%%%================================================================
%%%================      new commands      =======================
%%%================================================================

% \newcommand\hcancel[2][black]{\setbox0=\hbox{$#2$}%
% \rlap{\raisebox{.45\ht0}{\textcolor{#1}{\rule{\wd0}{1pt}}}}#2} 



%
%%% ---------------------- Referencing-------------------------

% Don't need those if \usepackage{cleveref} --> \cref{xxx}
% \newcommand{\tabr}[1]{Table~\ref{#1}}
% \newcommand{\eqr}[1]{Eq.~(\ref{#1})}
% \newcommand{\fir}[1]{Fig.~\ref{#1}}
% \newcommand{\secr}[1]{Sec.~\ref{#1}}
% \newcommand{\apr}[1]{App.~\ref{#1}}
% \newcommand{\chr}[1]{Ch.~\ref{#1}}

% % % Theorem environments
% % 
% \newtheorem{theorem}{Theorem}
% \newtheorem{prop}{Proposition}
% \newtheorem{definition}{Definition}
% \newtheorem{lemma}{Lemma}
% \newtheorem{protocol}{Protocol}
% \newtheorem{conjecture}{Conjecture}
% % 
% \newtheorem{cor}{Corollary}
% \newtheorem*{remark}{Remark}

%%% -------------------------- MATLAB, code and TODO environment
\newcommand{\matlab}{MATLAB}
\newcommand{\code}[1]{\texttt{#1}}
% \newcommand{\todo}[1]{\textcolor{red}{\textbf{TODO:}} \emph{\color{red}#1}}
% makes the whole column texttt style
\newcolumntype{C}{>{\collectcell\code}c<{\endcollectcell}}

% 
% %%% ---------------------- Custom Colors-------------------------
% 
\definecolor{rwthBlue100}{cmyk}{1,0.5,0,0}
\definecolor{rwthBlue75}{cmyk}{0.75,0.38,0,0}
\definecolor{rwthBlue50}{cmyk}{0.45,0.14,0,0}
\definecolor{rwthBlue25}{cmyk}{0.23,0.07,0,0}
\definecolor{rwthBlue10}{cmyk}{0.09,0.03,0,0}
\definecolor{DarkGray}{gray}{0.25} 
\definecolor{MidGray}{gray}{0.38} 
\definecolor{NeutralGray}{gray}{0.5}
\definecolor{LightGray}{gray}{0.7}
\definecolor{lightGray}{gray}{0.85}
\definecolor{DarkRed}{rgb}{0.7,0,0}
\definecolor{DarkBlue}{rgb}{0,0,0.5}
\definecolor{SteelBlue}{rgb}{0,0.4,0.6}
\definecolor{Orange}{rgb}{0.7,0.5,0}
\definecolor{Violette}{rgb}{0.5,0,0.5}
\definecolor{Sand}{rgb}{0.84,0.8,0.55}
\definecolor{niceblue}{rgb}{0.33,0.5,0.8}
\definecolor{OliveGreen}{RGB}{0,102,102}
\definecolor{NiceGreen}{RGB}{0,153,72}
\definecolor{LightGreen}{RGB}{0,200,72}
\definecolor{newblue}{RGB}{40,210,251}
\definecolor{lightblue}{RGB}{179,231,251}
\definecolor{steelblue}{RGB}{70,130,180}
\definecolor{cred}{RGB}{179,28,28}
% listings color
\definecolor{mygreen}{RGB}{28,172,0} % color values Red, Green, Blue
\definecolor{mylilas}{RGB}{170,55,241}
\usepackage{tcolorbox}
% 
% \tcbset{
%     coltitle=black,fonttitle=\bfseries,
%     colback=orange!7!white,colframe=gray!25!white,width=0.95\textwidth,valign=center,
%     before=\par\bigskip\centering,after=\par
%     }
% 
% %%% --------------------- Vector Notation -------------------
\newcommand{\uline}[1]{\underline{#1}}
\newcommand{\uuline}[1]{\underline{\underline{#1}}}
% 
% %%% ---------------------- Operators -------------------------
% %Use either \DeclareMathOperator or \operatorname http://tex.stackexchange.com/questions/84302/what-is-the-difference-of-mathop-operatorname-and-declaremathoperator
% 
% \DeclareMathOperator{\id}{id}
% \DeclareMathOperator{\tr}{tr}
% \DeclareMathOperator{\rg}{range}
% \DeclareMathOperator{\supp}{supp}
% \DeclareMathOperator{\sign}{sign}
% \DeclareMathOperator{\im}{im}
% \DeclareMathOperator{\poly}{poly}
% \DeclareMathOperator{\var}{Var}
% \DeclareMathOperator{\diag}{diag}
% \DeclareMathOperator{\eig}{eig}
% \DeclareMathOperator{\rank}{rank}
% \DeclareMathOperator{\cov}{cov}
% \DeclareMathOperator{\dist}{d}
% \DeclareMathOperator{\spec}{spec}
% \DeclareMathOperator{\trunc}{\upharpoonright}
% 
\newcommand{\mr}[1]{\mathrm{#1}}
\newcommand{\rmd}{\mathrm{d}}
\renewcommand{\dd}{\,\mathrm{d}}
% \DeclareMathOperator{\landauO}{O}
% % \newcommand{\trunc}[2]{#1_{\upharpoonright \mathnormal{#2}}} %truncation
% \DeclareMathOperator{\cc}{c.c.}
% \DeclareMathOperator{\hc}{h.c.}
%  
% %%% ---------------------- Others-------------------------
% \newcommand{\ad}{^\dagger}
% \newcommand{\restr}{\upharpoonright}
% \DeclareMathOperator{\dunion}{\uplus}%Disjoint union
% \DeclareMathOperator{\dUnion}{\biguplus}%Disjoint union
% \renewcommand{\complement}{^{c}}
% \DeclareMathOperator{\colonequiv}{:\!\Leftrightarrow}
% 
% \newcommand{\iprod}[2]{( #1, #2)}
% \newcommand{\Iprod}[2]{\left( #1, #2 \right)}
% 
% \newcommand{\set}[1]{\{ #1  \}}
% \newcommand{\Set}[1]{\left \{ #1 \right \}}
% 
\newcommand{\dt}[1]{\frac{\partial #1}{\partial t}}
\newcommand{\diff}[2]{\frac{\partial #1}{\partial #2}}
\newcommand{\ddiff}[2]{\frac{\partial^2 #1}{\partial #2^2}}
\newcommand{\Ddiff}[3]{\frac{\partial^2 #1}{\partial #2 \partial #3}}
% 
% 
% 
% %%% ---------------------- brakets -----------------
% %\usepackage{braket}
% \renewcommand{\l}{\langle}
% \renewcommand{\r}{\rangle}
% \newcommand{\Ket}[1]{\left|{#1}\right\rangle}
% \newcommand{\ket}[1]{|{#1}\rangle}
% \newcommand{\Bra}[1]{\left\langle{#1}\right|}
% \newcommand{\bra}[1]{\langle{#1}|}
% \newcommand{\braket}[2]{\left\langle{#1}\middle|{#2}\right\rangle}
% \newcommand{\ketn}[1]{| #1 \rangle}
% \newcommand{\bran}[1]{\langle #1 |}
% \newcommand{\braketn}[2]{\langle #1 \middle| #2 \rangle}
% \newcommand{\ketbra}[2]{\ket{#1} \!\! \bra{#2}}
% \newcommand{\ketbran}[2]{\ketn{#1} \! \bran{#2}}
% \newcommand{\proj}[1]{\ket{#1}\bra{#1}}
% \newcommand{\Proj}[1]{\Ket{#1}\Bra{#1}}
% \newcommand{\Sandwich}[3]
%   {\left\langle  #1 \right| #2 \left| #3 \right\rangle}
%   
% \newcommand{\sandwich}[3]
%   {\langle  #1 | #2 | #3 \rangle}
%   
% \newcommand{\Av}[1]{\left\langle #1 \right\rangle}
% \newcommand{\av}[1]{\langle #1 \rangle}
% \newcommand{\avb}[1]{\bigl\langle #1 \bigr\rangle}
% \newcommand{\avB}[1]{\Bigl\langle #1 \Bigr\rangle}
% 
% %%% ----------------- norms ----------------------
% \newcommand{\Norm}[1]{\left\Vert #1\right\Vert}
% \newcommand{\norm}[1]{\Vert #1\Vert}
% \newcommand{\normt}[1]{\boldsymbol{\vert} #1\boldsymbol{\vert}}
% \newcommand{\normone}[1]{\| #1 \|_1}
% \newcommand{\norminf}[1]{\| #1 \|_\infty}
% \newcommand{\normuinv}[1]{||| #1 |||}
% \newcommand{\hnorm}[1]{||| #1 |||}
% \newcommand{\normb}[1]{\bigl\| #1 \bigr\|}
% \newcommand{\normB}[1]{\Bigl\| #1 \Bigr\|}
% \newcommand{\abs}[1]{\lvert #1\rvert}
% \newcommand{\Abs}[1]{\left\lvert #1\right\rvert}
% 
% 
% %%% ---------------------- Letters-----------------
% 

% \newcommand{\ee}{\mathrm{e}}
% \newcommand{\ii}{\mathrm{i}}
% \newcommand{\dd}{\mathrm{d}}
% 
% 
% \newcommand{\GG}{\mathbf{G}}
% \newcommand{\LL}{\mathbf{L}}
% \newcommand{\VV}{\mathbf{V}}
% 
% 
% \newcommand{\qb}{\mathbf{q}}
% \newcommand{\pb}{\mathbf{p}}
% \newcommand{\rb}{\mathbf{r}}
% \renewcommand{\sb}{\mathbf{s}}
% \newcommand{\tb}{\mathbf{t}}
% \newcommand{\xb}{\mathbf{x}}
% \newcommand{\yb}{\mathbf{y}}
% 
% 
% \newcommand{\jc}{\mathcal{j}}
% \newcommand{\Ac}{\mathcal{A}}
% \newcommand{\Bc}{\mathcal{B}}
% \newcommand{\Cc}{\mathcal{C}}
% \newcommand{\Dc}{\mathcal{D}}
% \newcommand{\Ec}{\mathcal{E}}
% \newcommand{\Fc}{\mathcal{F}}
% \newcommand{\Gc}{\mathcal{G}}
% \newcommand{\Hc}{\mathcal{H}}
% \newcommand{\Ic}{\mathcal{I}}
% \newcommand{\Jc}{\mathcal{J}}
% \newcommand{\Kc}{\mathcal{K}}
% \newcommand{\Lc}{\mathcal{L}}
% \newcommand{\Mc}{\mathcal{M}}
% \newcommand{\Nc}{\mathcal{N}}
% \newcommand{\Oc}{\mathcal{O}}
% \newcommand{\Pc}{\mathcal{P}}
% \newcommand{\Qc}{\mathcal{Q}}
% \newcommand{\Rc}{\mathcal{R}}
% \newcommand{\Sc}{\mathcal{S}}
% \newcommand{\Tc}{\mathcal{T}}
% \newcommand{\Vc}{\mathcal{V}}
% \newcommand{\Uc}{\mathcal{U}}
% \newcommand{\Wc}{\mathcal{W}}
% \newcommand{\Xc}{\mathcal{X}}
% \newcommand{\Yc}{\mathcal{Y}}
% 
% \newcommand{\Ab}{\mathbb{A}}
% \newcommand{\Cb}{\mathbb{C}}
% \newcommand{\Eb}{\mathbb{E}}
% \newcommand{\Lb}{\mathbb{L}}
% \newcommand{\Nb}{\mathbb{N}}
% \newcommand{\Pb}{\mathbb{P}}
% \newcommand{\Qb}{\mathbb{Q}}
% \newcommand{\Rb}{\mathbb{R}}
% \newcommand{\Tb}{\mathbb{T}}
% \newcommand{\Zb}{\mathbb{Z}}
% \newcommand{\idb}{\mathbb{1}}
% 
% 
% %%%%%%%%%%%%%%%%%%%%%%%%%%%%%%%%%%%%%%%%%%%%%%%%%%%%%%%%%%%%%%%%%%%5
% % Specific to this project
% %%%%%%%%%%%%%%%%%%%%%%%%%%%%%%%%%%%%%%%%%%%%%%%%%%%%%%%%%%%%%%%%%%%5
% 
% % Complexity classes
% \newcommand{\sat}{\textsf{SAT}}
% \newcommand{\qsat}{\textsf{QSAT}}
% 
% \newcommand{\p}{\textsf{P}}
% \newcommand{\bpp}{\textsf{BPP}}
% \newcommand{\np}{\textsf{NP}}
% 
% \renewcommand{\sp}{\#\textsf{P}}
% 
% \newcommand{\bqp}{\textsf{BQP}}
% \newcommand{\qma}{\textsf{QMA}}
% \newcommand{\ma}{\textsf{MA}}
% 
% %%% -------------------------- Commments ----------------------------
% 
% \newcommand\contribution[1]{\emph{\textcolor{gray!40!black}{#1}}}
% \newcommand{\ins}[1]{\textcolor{cred}{*** #1 ***}}

