\chapter*{Abstract}
\todo{This is still the wrong abstract}

In recent years, the experimental capability in realizing and exercising full
control over multiple coupled quantum dots in semiconductor heterostructures
has advanced to the point where tuning the dots to the few-electron regime
required for quantum computation applications will become impractical to
perform manually, establishing the demand for automated procedures in this
area. Moreover, charge stability diagrams, which represent a crucial tool in
the tuning of quantum dot systems, become increasingly complex to understand
and analyze due to the growing number of parameters influencing their
structure. The goal of this thesis is to take a first step towards a fully
computer-automated implementation of the tuning process by focusing on the
coarse tuning of the system to the zero-electron regime.

I simulate charge diagrams of multiple coupled quantum dots in a linear array
using a classical capacitive model, reproducing the honeycomb structure of
double quantum dot charge diagrams. In order to automatically recognize the
zero-electron regime in double-dot charge diagrams by identifying the charge
transitions, I introduce a line detection algorithm using the Radon transform
of the charge diagram able to detect lines even in noisy images by exploiting
the fact that, in a suitable charge diagram detail, several transitions of the
same kind and hence the same slope will show. The strong noise performance of
the algorithm promises fast scan times during the tuning process. Finally, I
present and test an algorithm that attempts to automatically tune a simulated
double-dot system to the zero-electron regime using the line detection
algorithm developed.