In order to understand the complexity of the problem \todo{justification for
    modelling with elastic waves} and the
utilised solving method, it is reasonable to revisit some fundamental concepts
in the following chapter. At the same time, a consistent notation is
introduced. At first, the tensor formalism for elastic properties
of rigid bodies is introduced. It should be noted in advance, that vectors are
denoted by a single underline and tensors or
tensor fields of higher order by a number of underlines corresponding to their
order. \\ After that, elastic waves are derived and analysed
further. This will be mostly based on \cite{GrossMarx2014}, if not otherwise
stated.
\todo{check if description still applies, citation?}

\section{Elastic Properties of Materials}
In the following, the formalism for bulk elastic properties of materials is
introduced. As only small deformations are relevant for the considerations in
this thesis, they are assumed to be in the regime of linear elastic behaviour,
neglecting plastic or nonlinear behaviour.

In one dimension this can be described simply by Hooke`s law, which states
that a small displacement $\Delta l$ caused by a force $F$ is
proportional to the displacement. Considering an object of length $l$ and
cross-sectional area $A$, \textbf{stress} and \textbf{strain} can be defined as

\begin{align}
     & \text{Stress} \quad \quad \sigma := \frac{F}{A}
     & \text{Strain} \quad \quad \epsilon := \frac{\Delta l}{l}
\end{align}

Hooke`s law can now be reformulated as
\begin{equation}
    \label{eq:HookStress1D}
    \sigma = C\ \epsilon
\end{equation}
with \textbf{Young`s Modulus C} as proportional constant.

\subsection{Stress Tensor}
For finite size objects stress can be defined locally on
infinitesimal, cubic volume elements. These volume elements get deformed by
forces that are applied to the object. If we consider a general force
$\Delta \underline{F}$ acting on a surface element $\Delta A$ we can always
divide
this force into a normal component $\Delta \underline{F}_n$ and two mutually
perpendicular tangential components $\Delta \underline{F}_{t1}$ and
$\Delta \underline{F}_{t2}$. This implies the general definition of the stress
tensor
as
\begin{equation}
    \sigma_{ij} = \frac{\text{force in direction i}}{\text{surface with normal
            in direction j}}
\end{equation}
\todo{picture of stress definition?}
whereas indices i and j denote denote one of the spatial directions x,y or z.
If we claim the volume element to be static, it follows, that the normal forces
on opposite sites and tangential forces on neighbouring sides equal each other.
The latter can be expressed over $\sigma_{ij}=\sigma_{ji}$. This leaves 6
independent components, the three normal stresses $\sigma_{ii}$ and the three
shear stresses $\sigma_{ij}$.

\subsection{Strain Tensor}
Deformation of a three dimensional object can be described using the
displacement field $\underline{u}(\underline{r}, t):=
    \underline{r}^\prime-\underline{r}$,
which defines a displacement vector for each point $\underline{r}$ in space in
comparison to the deformed position $\underline{r}^\prime$. Local stress only
relates to change in displacement relative to neighbouring positions which
allows to consider the Taylor expansion in first order
\begin{equation}
    \underline{u}(\underline{r}+\Delta\underline{r}, t)
    = \underline{u}(\underline{r}, t) +
    \underline{\underline{\gradient}}u(\underline{r}, t)\ \Delta \underline{r}
\end{equation}
with the Jacobian matrix $(\underline{\underline{\gradient}}u )_{ij} =
    \frac{\partial u_i}{\partial r_j}$. This in turn can be decomposed into a
symmetric part and an antisymmetric part. The symmetric part is defined as
strain tensor:
\begin{equation}
    \epsilon_{ij} = \frac{1}{2} \left(	 \frac{\partial u_i}{\partial r_j}
    +\frac{\partial u_j}{\partial r_i} \right)
\end{equation}
It is a dimensionless measure for local deformation in contrast to the
antisymmetric part, which represents local rotation.
\todo{reference to lecture 2?}

\subsection{Stress-Strain Relations}
With stress and strain tensor introduced, equation \ref{eq:HookStress1D} can
be generalised to three dimensions taking anisotropies of the material into
account:
\begin{align} \label{eq:HookStress3D}
     & \sigma_{ij} =  C_{ijkl}\epsilon_{kl} \quad\quad\quad
    \quad \quad \text{or}
     & \uuline{\sigma} = \uuline{\uuline{C}} \cdot \uuline{\epsilon}
\end{align}
This defines the \textbf{Elasticity tensor} as order 4 tensor with unit force
/ area. In general this tensor contains 81 components. However, stress and
strain tensor are symmetric, so that $C_{ijkl} = C_{jikl} = C_{ijlk}$. This
reduces the number of independent components to 36 and makes it possible to
describe the Elasticity tensor as $6\cross6$ matrix. For this representation,
the so called Voigt notation maps pairs of coefficients of elasticity, strain
and stress tensor to a single index as in table \ref{tab:voigt}
\begin{table}[h]
    \centering
    \begin{tabular}{cccccc}
        $xx \rightarrow 1$ & $yy \rightarrow 2$    & $zz \rightarrow 3$    &
        $yz=zy
        \rightarrow 4$     & $xz=zx \rightarrow 5$ & $xy=yx \rightarrow 6$
    \end{tabular}
    \caption{Voigt notation}
    \label{tab:voigt}
\end{table}
This simplifies the notation of elasticity significantly, but it should be
treated carefully. Tensors written in Voigt notation do not transform like
vectors in each index.

One can also proof, that $C_{ijkl} = C_{klij}$ considering elastic energy (see
section 4.3.1 of \cite{GrossMarx2014}) which leads to 21 independent
components. Further simplifications derive from crystal symmetry or
limitations in anisotropy. In the following treatment, the material is assumed
to be isotropic, which reduces the number of independent components to 2. These
are typically introduced as Lamé constants $\lambda$ and $\mu$ defined by the
relation
\begin{equation} \label{eq:StressStrainIso_tensor}
    \sigma_{ij} = \lambda \delta_{ij} \epsilon_{kk} + 2\mu \epsilon_{ij}
\end{equation}
\todo{mention of einstein convention?}
which derives from equation \ref{eq:HookStress3D} by regarding isotropy of the
material \cite{kundu2012ultrasonic}. From this equation the elasticity tensor
can be expressed by a simplified $6\cross6$ matrix as in:
\begin{equation} \label{eq:StressStrainIso}
    \uuline{\sigma} = \uuline{\uuline{C}}\cdot \uuline{\epsilon} =
    \begin{pmatrix}
        \lambda + 2\mu & \lambda        & \lambda        & 0    & 0    & 0
        \\
        \lambda        & \lambda + 2\mu & \lambda        & 0    & 0    & 0
        \\
        \lambda        & \lambda        & \lambda + 2\mu & 0    & 0    & 0
        \\
        0              & 0              & 0              & 2\mu & 0    & 0
        \\
        0              & 0              & 0              & 0    & 2\mu & 0
        \\
        0              & 0              & 0              & 0    & 0    & 2\mu
        \\
    \end{pmatrix}
    \cdot
    \begin{pmatrix}
        \epsilon_1 \\ \epsilon_2 \\ \epsilon_3 \\ \epsilon_4 \\ \epsilon_5 \\
        \epsilon_6 \\
    \end{pmatrix}
\end{equation}

\section{Elastic Waves}
This formalism can now be used to introduce elastic waves. Their existence and
behaviour depends on the equations of motion of the regarded medium. To obtain
these, we consider at first a small volume $\Delta V = \Delta x \Delta y
    \Delta z$, that is subject to a stress $\sigma_{xx}(x)$ on the one side and
to $\sigma_{xx}(x+\Delta x)$ on the other side. The resulting net force becomes
thus
\begin{equation}
    \Delta F_x = \left[ \sigma_{xx}(x+\Delta x)- \sigma_{xx}(x) \right] \Delta
    x \Delta y = \diff{\sigma_{xx}}{x} \Delta x \Delta y \Delta z
\end{equation}
by approximating $\sigma_{xx}(x+\Delta x)$ in first order and setting the frame
of reference to the center of mass of the volume.
The force leads to a displacement $u_x$ of the volume in x direction and equals
the
product of the mass $\rho\Delta x \Delta y \Delta z$ and acceleration in x
direction $\ddiff{u_x}{t}$. This yields the one dimensional partial
differential equation
\begin{equation}
    \rho \ddiff{u_x}{t} = \diff{\sigma_{xx}}{x}
\end{equation}
Assuming an isotropic material and using relation \ref{eq:HookStress3D} we
get
\begin{equation}
    \rho \ddiff{u_x}{t} = C_{11} \diff{\epsilon_{xx}}{x} = C_{11}\ddiff{u_x}{x}
\end{equation}
In an anisotropic medium the other stress components need to be taken into
account leading to the general wave equation
\begin{equation} \label{eq:wave_eq}
    \rho \ddiff{u_i}{t} = \diff{\sigma_{ij}}{x_j} = C_{ijkl}
    \Ddiff{u_l}{x_j}{x_k}
\end{equation}
% In analogy to electromagnetic waves we can now make the harmonic plane wave
% ansatz $\uline{u}(\uline{x}, t)= \uline{u}_0\ e^{i(\uline{k}\uline{x}-\omega
%             t)}$. At this, the vector $\uline{k}=\frac{2\pi}{\lambda}$ denotes
% the wave vector defined over the wavelength $\lambda$. $\omega$ is the circular
% frequency defined over the  oscillation frequency $f$ with  $\omega = 2 \pi f$.

% As this solution can be understood as a single frequency mode in Fourier
% transformation, one can achieve arbitrary wave shapes by transforming a mode
% distribution $\uline{u}_0 = \uline{u}_0(\uline{k})$ via Fourier transformation.
% Such an arbitrary wave will behave according to the behaviour of
% the individual plane waves which will be discussed in the following.

% In case of an anisotropic medium one would proceed by substituting the ansatz
% into equation \ref{eq:wave_eq} and

\subsection{Solution for Isotropic Materials}
\label{sec:IsoSolution}
In the following analysis, only isotropic media will be considered. In this
case the isotropic stress strain relation, equation
\ref{eq:StressStrainIso}, can be inserted to the wave equation which
yields
\begin{equation}
    \rho \ddiff{\uline{u}}{t} = (\lambda + \mu)\ \uline{\gradient}
    (\uline{\gradient}\cdot \uline{u}) + \mu\ \uline{\gradient}^2 \uline{u}
\end{equation}
By using the relation $\gradient^2 \uline u = \uline{\gradient}
    (\uline{\gradient}\cdot \uline{u}) - \uline{\gradient}
    \cross(\uline{\gradient} \cross \uline{u})$ for the vector laplace operator
one can express it as
\begin{equation} \label{eq:balanceIso}
    \rho \ddiff{\uline{u}}{t} = (\lambda + 2\mu)\ \uline{\gradient}
    (\uline{\gradient}\cdot \uline{u})- \mu\ \uline{\grad }
    \cross (\uline{\gradient}\cross \uline{u})
\end{equation}
If we now consider the Helmholtz decomposition of the displacement fields
\begin{equation} \label{eq:Helmholtz}
    \uline{u} = \uline{\gradient}\Phi + \uline{\gradient}\cross \uline{\Psi}
\end{equation}
with elastic potentials $\Phi$ and $\Psi$ and insert it into
\ref{eq:balanceIso},
it decouples to the two equations (see \cite{BedfordElasticWaves})
\begin{align}
    \ddiff{\Phi}{t} = \alpha^2 \uline{\gradient}^2\Phi \\
    \ddiff{\uline{\Psi}}{t} = \beta^2 \uline{\gradient}^2\uline{\Psi}
\end{align}
with $\alpha = \left(\frac{\lambda + 2\mu}{\rho}\right)^{1/2}$ and $\beta =
    \left(\frac{\mu}{\rho}\right)^{1/2}$. Those equations represent four wave
equations for the elastic potentials with phase velocities $\alpha$ and
$\beta$.

In analogy to electromagnetic waves we can now make the harmonic
plane wave ansatz $\Phi(\uline{x}, t)= u_L\
    e^{i(\uline{k}_L \uline{x}-\omega t)}$ and $\uline{\Psi}(\uline{x}, t)
    =
    \uline{u}_T \ e^{i(\uline{k}_T\uline{x}-\omega t)}$ with arbitrary
amplitudes $u_L\in\mathbb{C}$ and $\uline{u}_T\in \mathbb{C}^3$.
At this, the vector
$\uline{k}=\frac{2\pi}{\lambda}$ denotes the wave vector defined over the
wavelength $\lambda$ and $\omega$ is the circular frequency defined over the
oscillation frequency $f$ with	$\omega = 2 \pi f$. The differentiation between
$\uline{k}_L$ and $\uline{k}_T$ is necessary because of the different
phase velocities for the elastic potentials resulting in different wave numbers
according to $c = \frac{\omega}{|\uline{k}|}$.

Substituting the wave ansätze into equation \ref{eq:Helmholtz}, we get
\begin{equation}
    \uline{u} = u_L \uline{k}\	e^{i(\uline{k}_L\uline{x}-\omega t)}
    + \uline{k} \cross \uline{u}_T\ e^{i(\uline{k}_T\uline{x}-\omega t)}
\end{equation}
Here we can see now, that the elastic potential $\Phi$ is responsible for waves
with longitudinal polarisation and $\uline{\Psi}$ for waves with transversal
propagation. Furthermore we can identify an orthonormal polarisation basis
depending on the propagation direction $\uline{\hat{k}} =
    \frac{\uline{k}}{|\uline{k}|}$ so that $\uline{u}$ decomposes to
\begin{equation}
    \uline{u} = a_L \ \uline{p}_L\	e^{i(\uline{k}_L\uline{x}-\omega
            t)}
    + (a_{TH}\ \uline{p}_{TH} + a_{TV}\ \uline{p}_{TV})\
    e^{i(\uline{k}_T\uline{x}-\omega t)}
\end{equation}
In gerneral, the polarisation vector for longitudinal polarisation
$\uline{p}_L$ is fixed to be $\uline{\vec{k}}$. If scattering at an interface
is considered, a plane of incidence can be defined that is spanned by
$\uline{\hat{k}}$ and the normal vector of the interface. The transversal
polarisation vectors are then defined as transversal horizontal polarisation
$\uline{p}_{TH}$ pointing out of the plane of incidence and transversal
vertical polarisation lying in plane of incidence orthonormal to
$\uline{\hat{k}}$.

The coefficients $a_i$ may be complex to express an additional phase between
the components. However, it should be noted, that this is merely convenient for
calculation and that the physical wave behaves like the real part of the shown
equations.

% \section{Elastic Waves in Stratified Media}
% The previous considerations enable now to target the description of elastic
% wave propagation through layer structures, for instance distributed bragg
% reflectors. Those structures consist of homogeneous materials with different
% mechanical properties and thus phase velocities per material. 
% In the following the layers
% are assumed to be parallel to the xy-plane and have a defined thickness in
% z-direction. In x and y-direction they are infinitely stretched out, which is a
% good approximation, if the thickness is small compared to the width.
% % 
% \todo{alternative: reflecting boundaries on the side}
% % 

\subsection{Elastic Wave Scattering at an Interface}
%Zwischentext einfügen ! #################################################################

\subsubsection{Boundary Conditions}
On the way to describe wave propagation through complex layer structures it is
useful to consider a single interface first. In the upper half space $S_1$ and
lower half space $S_2$
we assume homogeneous materials with different elastic constants and thus
different sound velocities. At the interface, which is chosen to be the
xy-plane the elastic properties are discontinuous. However, we can assume that
those layers connected strongly so that the displacement field and normal
stresses are continuous at the boundary. This leads to the boundary conditions:
% 
\todo{lookup citation from \cite{brekhovskikh2012waves}}
% 
\begin{align} \label{eq:boundaryconditions}
    \uline{u}^{(1)}(\uline{r}, t)|_{\uline{r}\in\partial S_1}   & =
    \uline{u}^{(2)}(\uline{r}, t)|_{\uline{r}\in\partial S_2}       \\[5pt]
    \sigma_{13}^{(1)}(\uline{r}, t)|_{\uline{r}\in\partial S_1} & =
    \sigma_{13}^{(2)}(\uline{r}, t)|_{\uline{r}\in\partial S_2}     \\[5pt]
    \sigma_{23}^{(1)}(\uline{r}, t)|_{\uline{r}\in\partial S_1} & =
    \sigma_{23}^{(2)}(\uline{r}, t)|_{\uline{r}\in\partial S_2}     \\[5pt]
    \sigma_{33}^{(1)}(\uline{r}, t)|_{\uline{r}\in\partial S_1} & =
    \sigma_{33}^{(2)}(\uline{r}, t)|_{\uline{r}\in\partial S_2}
\end{align}
In these six equations displacement and stress from upper and lower medium are
differentiated by the given superscript index.

\subsubsection{Law of Refraction}
% spatial orientation
If we now consider a plane wave that is incident on that interface, we can
define a plane of incidence as in section \ref{sec:IsoSolution} and rotate the
coordinate system so that it equals the yz-plane. This is only valid if we
assume an isotropic material as stated before.
In that setting, the wavevector $\uline{k}_i$ for a particular mode $i \in
    \{L,TV,TH\}$ can be parametrised by the angle of incidence $\theta$ as
\begin{equation}
    \uline{k}_i = \frac{\omega}{c_i}\, ( 0,\ \sin(\theta_i),\ -\cos(\theta_i)\
    )^T
\end{equation}
% polarisation basis
The polarisation basis can then be defined as
\begin{align}
    \uline{p}_L = \uline{\hat{k}} & = ( 0,\ \sin(\theta_i),\ -\cos(\theta_i)\
    )^T
    \\
    \uline{p}_{TH}                & = (\ 1,\ 0,\ 0)^T
    \\
    \uline{p}_{TV}                & = \uline{p}_L \cross \uline{p}_{TH}
    = ( 0,\ -\cos(\theta), -\sin(\theta))^T
\end{align}
% general snellius
It is now possible to extract a law of refraction from the boundary conditions
in equation \ref{eq:boundaryconditions} (see \cite{achenbach1973wave}, pp.168ff). 
A simple approach is to consider
scattering of the transversal horizontal mode (TH) so that the displacement is
only in x-direction. An incoming plane wave with
wave vector $\uline{k}_{in}$ is partially reflected to a wave along
$\uline{k}_r$ and partially transmitted through the interface to a wave with
$\uline{k}_t$. Also, different frequencies $\omega_i$ are assumed for the
scattered waves so that the wave in the upper medium is
\begin{equation}
    u_1^{(1)} = a_{in}\
    e^{i( k_{in}\sin\theta_{in} y-\omega_{in} t)}
    + a_r\ e^{i( k_{r}(\sin\theta_{r} y -\omega_{r} t)}
\end{equation}
and in the lower medium
\begin{equation}
    u_1^{(2)} = a_{t}\
    e^{i(( k_{t}\sin\theta_{t} y -\omega_{t} t)}
\end{equation}
Here it was used, that the interface is at $z=0$ which simplifies the following
derivation.
According to the boundary conditions, these displacements must equal each other
at any time and for all points on the interface. This leads to the immediate
conclusion, that
\begin{equation} \label{eq:equalfreq}
    \omega = \omega_{in} = \omega_r = \omega_t
\end{equation}
After removing the common factor $e^{-i\omega t}$ from the equation and
expressing the wave numbers by frequency and sound velocity over
$k = \frac{\omega}{c}$, we get
\begin{equation} \label{eq:snelliuscompare}
    a_{in}\ e^{i \frac{ \omega}{c_{T,1}}\sin\theta_{in} y}
    + a_{r}\ e^{i \frac{ \omega}{c_{T,1}}\sin\theta_{r} y}
    = a_{t}\ e^{i \frac{ \omega}{c_{T,2}}\sin\theta_{t} y}
\end{equation}
This however can be only fulfilled vor all $y \in \mathbb{R}$, if the relations
\begin{align}
    a_{in} + a_r = a_t \\
    \frac{\sin\theta_{in}}{c_{T,1}} = \frac{\sin\theta_{r}}{c_{T,1}} =
    \frac{\sin\theta_{t}}{c_{T,2}} \label{eq:gensnell}
\end{align}
are correct. Equation \ref{eq:gensnell} can be also interpreted as the equality
of the component parallel to the interface of wave vector
$k_y=\sin\theta_i \frac{\omega}{c_i}$.

The same procedure can be applied to the y and z component of the displacement
field, where transversal vertical ($TV$) and longitudinal ($L$) modes need to
be considered. However, one significant difference is, that even for a single
incident mode both $L$ and $TV$ mode are possible after being reflected or
transmitted. This will only add an additional addend of the form
$ a_{i}\ e^{i \frac{ \omega}{c}\sin\theta_{i} y}$ to each side of equation
\ref{eq:snelliuscompare}, but will result in analoguous relations to
\ref{eq:equalfreq} and \ref{gensnell}, namely the invariance of frequency
$\omega$ and $k_2$ of each mode under scattering at the interface.

It is also possible to get the analytical solutions for refraction of a single
mode at an interface by evaluating the remaining boundary conditions and the
results from continuity of displacement.
% algebraic solution to single interface?

% energy partition at single surface

\subsection{Solving Methods for Stratified Media}

\subsubsection{Transfer Matrix Method}

\subsubsection{Linear System of Equations}