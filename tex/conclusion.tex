To conclude, a working simulation framework could be developed. In the course
of developement, the numerical issues previously encountered by Miles could be
mostly removed and the functionality of the calculation over all angles was
established. Several kinds
of validation, like the check for energy conservation or relations at a single
interface reason confidence in the correctness of all not directly verified
computations. Also, the sanity check shows to be very capable of detecting
numerical instabilities without larger effort.

Although these capabilities are enough to fully calculate larger systems,\\
\ttt{TransferMethodMP()} is multiple times slower than \ttt{SingleLSEMethod()}
so that it may be of interest to optimise the methods calculating the
transmission. One way could be, to take advantage of packages like \ttt{numba}
that are able to speed up python code. In that case, it is necessary to
reimplement the optimised method without usage of custom functions or objects
except for \ttt{numpy} types as these can not be handled by \ttt{numba}.

In the future, this simulation can be used to develop a structure that is able
to decrease the total expected heat flux to few mW. The theoretical foundations
for a systematical search were given in this work. One could think about using
the knowledge of the behaviour of the spectral bandgaps for normal incidence
to stack several reflectors to a fully reflecting structure. If this can be
achieved for normal incidence, the intensity for incident waves for higher
angles will be reflected at high rates as well. However, to obtain certainty it
must be validated by simulation.

The selection of the right materials will also be a relevant question, which
can be also conducted by the given knowledge for the reflection at single
interfaces. Some of those ideas are also regarded in the Jupyter notebook
\ttt{dbr\_search.ipynb}

Ultimately, one can fabricate such a structure and then measure its thermal conductance
to verify the expectations. In that setting it would be of interest to investigate the
influence of surface roughness of the interfaces on the coherence of the
phonons. The possibility of surface roughness could also be implemented into
the existing model in a future work by following the generalised transfer matrix method
presented in \cite{Katsidis2002}.