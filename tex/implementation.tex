On basis of the preceding theory a full simulation of acoustic wave propagation
through multilayer structures was developed. This will be
described in more detail in the following aiming to provide a short
introduction into code structure and functionality. Arisen challenges and their
solution will be presented afterwards.

\section{Code Structure and Functionality}
The main components of the project are represented by the three python modules
\texttt{dbr.py}, \texttt{materials.py} and \texttt{plotfunctions.py} to
organise written methods by purpose.

\texttt{materials.py} contains elastic parameters of materials that where
considered for usage in a reflector structure. Each entry is a python function
that returns density $\rho$ and elastic constants $\lambda$ and $\mu$. Most
databases accessible through the internet contain different elastic constants
like Young`s Modulus or Poisson`s ratio, so that those are converted with help
of a formula table as in \cite[30]{kundu2012ultrasonic} \todo{Formeln in
    Theorieteil?}. The exact materials and their sources are listed in table
\todo{reference to material parameters}

The file \texttt{dbr.py} contains the most important simulation tools. Those
functions are structured in two classes, \texttt{Reflector} and \texttt{Layer}
to reduce the number of parameters to be passed to the next method.

The main idea for the \texttt{Layer} class, is to wrap the important material
parameters $\lambda$, $\mu$ and $\rho$ passed from a material parameter
function and to convert them internally to sound velocities $c_i$ and elastic
tensor. In addition, each object stores its thickness and material name. The
latter is useful for fast identification of the layer material and debugging
outputs.
Also, it contains several methods for calculation of Layer specific
quantities. One of those quantities is \texttt{getConditionMatrix()} which
assembles the condition matrix $\uuline{M}_n$ for given incident angle
$\theta$ and frequency $\omega$. In the process, the layer specific wave vector
$\uline{k}_i$ is calculated by \texttt{Layer.getK()} for every mode from
incident angle and frequency. The phase velocities are accessed from the
object. Calculation of polarisation vectors (\texttt{getPolarisations()}) and
strain tensor (\texttt{getStrain()}) follows the same principle. The functions
\texttt{Intensity()} implementing equation \ref{eq:generalIntensity} and
\texttt{getPropagationMatrix()} implementing $\uuline{P}_n$ from equation
\ref{eq:PmatTMM} support the later mentioned methods from
\texttt{Reflector}.

To construct a layer structure that can be evaluated by the simulation tools of
\texttt{Reflector}, those \texttt{Layer} objects can be assembled in a list,
which indicates the order of Layers in the reflector structure. Together with
the defined thicknesses of each \texttt{Layer} object this forms an unambiguous
representation of a physical layer structure.

This list of \texttt{Layer} objects is then passed to the constructor of a
\texttt{Reflector} object and stored as deep copy of the initial list to avoid
bugs through references to the same object. Besides, the layer thicknesses are
extracted and stored internally in a separate list which is used exclusively
over the object stored thicknesses. The first and the last given
layer are assumed to embed the remaining layers, so that their thickness
defined in the \texttt{Layer} object has no meaning. For convenience in the
simulation functions, the thicknesses of first and last layer are set to $0$
in \texttt{Reflector.\_\_init\_\_()}. The thickness configuration can be
changed
with \texttt{setThicknesses()} and a full text representation of the current
configuration can be generated with \texttt{info()}.

\subsection{Algorithms and their Scope of Application }
There are four methods implemented to calculate the transmittivity function
$\mathcal{T}_i(\omega, \theta)$ differing in numerical stability and speed.
Those are \ttt{TransferMethod()} and \ttt{TransferMethodMP()}, which implement
the Transfer Matrix Method. On the other side \ttt{LSEMethod()} and
\ttt{SingleLSEMethod()} implement the described LSE method.
To explain their coexistence, it is helpful to describe them in the order of
their development.

The first attempt was \ttt{TransferMethod()}. The TMM offers the most promising
properties with regard to computation time for large systems, because existing
matrix mulitplication algorithms as implemented in the \ttt{numpy} package can
be used and the addition of layers to the problem only increases the
computation time linearly. The latter becomes apparent when considering
equation \ref{eq:Scomposition}.

The concrete implementation is as follows. \ttt{TransferMethod()} called with
arrays of circular frequencies $\omega$, incident angles $\theta$ and the
coefficient vector $t_0$ passes those parameters first to
\ttt{getTotalTransfermatrix()}. This function calculates the $\uuline{S}$
matrix for the entire system as in equation \ref{eq:Scomposition}. The function
iterates in a python \ttt{for}-loop over all defined interfaces. In each
iteration $n$, the transmitted angles from the previous interface
$\theta_{n-1,i}$
are used as incident angles on the current interface $n$. Those are needed to
calculate the wave vectors and the propagation matrix $\uuline{P}_n$ with
\ttt{getK()} and \ttt{getPropagationMatrix()} called on the upper Layer object.
$\uuline{S}_n$ is also calculated from $\theta_n$ and $\omega$ with
\ttt{TransferMatrix()}. Those two matrices, $\uuline{S}_n$ and $\uuline{P}_n$
are then multiplied to the evolving total S-matrix of the previous iteration 
$\uuline{S}_{n-1}^\prime$ as
\begin{equation}
    \uuline{S}_n^\prime = \uuline{S}_{n-1}
\end{equation}

\todo{concrete form of condition matrix im in appendix}

%BraggFreq