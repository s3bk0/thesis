In recent years quantum computing technology has been of rising
interest to scientists and the public. The early steps of exploiting the nature
of quantum states by fabricating and controlling qubits [refs] are
overcome and quantum supremacy, the supremacy of quantum computers over
classical computers, is already achieved [ref]. With scalable concepts for many
different implementations [refs] it is only a matter of time until quantum
computers can be used commercially.

However, there are still a some hurdles to overcome before the technology is at
this level. One of them is to reduce the decoherence time, which strongly
depends on external conditions like for example, the influence of thermal
radiation caused by photons or phonons[refs]. In solid state qubits, heat
transfer by
phonons is mostly enabled by electronics connected to the chip. These therefore
come from a known direction and thus enable the following solution approach.
Distributed Bragg reflectors known from optics as layer structures with
alternating refraction indices could be used as an intermediate component to
reflect the incoming phonons and thus shield the chip.

In optics, this structure is usually used as reflector for light from a
defined angle, where it forms spectral regions with high reflectivites. For our
purpose, the complete spectrum of thermal radiation needs to be reflected
regardless of incident angle in order to stop any incoming phonons. This will
be quantified later.

\begin{equation} \label{eq:totalheatflow}
    J_z = \sum\limits_{i=\{L,TV,TH\}}\ \int_0^{\omega_c}\int_{\theta, \phi}
    \mathcal{T}_i(\omega, \theta, \phi) \expval{n_B(\omega, T)} \hslash \omega\ 
    c_i\cos \theta \dd{\omega} \dd{\theta}\dd{\phi}
\end{equation}

\section{Assumptions}
\todo[inline]{Introduction to notation}
\begin{itemize}
    \item notation of vectors and tensors
\end{itemize}

\section{Outline}