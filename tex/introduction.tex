In recent years quantum computing technology has been of rising
interest to scientists and the public. The early steps of exploiting the nature
of quantum states by fabricating and controlling qubits [refs] are
overcome and quantum supremacy, the supremacy of quantum computers over
classical computers, is already achieved [ref]. With scalable concepts for many
different implementations [refs] it is only a matter of time until quantum
computers can be used commercially.

However, there are still a some hurdles to overcome before current setups are
at this level. One of them is to reduce the decoherence time. It is crucial for
the execution of large operations or algorithms and strongly
depends on external conditions like for example, the influence of thermal
radiation caused by photons or phonons[refs]. In solid state qubits, heat
transfer by phonons is mostly enabled by electronics connected to the chip.
This could be solved by a intermediate medium with very low thermal
conductivity shielding the chip from thermal phonons.

\section{Superlattice Structures for Thermal Isolation}
One possible approach to
construct such a medium is by developing a superlattice structure that
uses the mechanics of coherent phonon heat conduction. This concept exploits
the properties of coherent phonons as coherent elastic waves and has proven
to be effective for control of heat conduction at low temperatures
    [LuckyanovaGarg]. The underlying theory of this concept focuses on a
special case of the macroscopic theory of heat conduction, where heat flow is
explained as diffusive scattering of phonons between regions of different
temperatures.
However, the presented concept relies on interference effects of scattering
at the constructed interfaces, so that such structures are significantly
smaller than the phonon mean free path.

Further, such superlattices are known from optics as Distributed Bragg
reflectors and usually consist of layer structures with alternating refraction
indices. These are known to have so called spectral band gaps which are
spectral regions where the wave is fully reflected.
In optics this structure is usually used as highly efficient reflector for
electromagnetic waves from a defined angle.

The aim of this work will be to investigate practicable reflector
configurations by developing a reusable simulation of elastic wave propagation
through arbitrary multilayer structures. With this tool some configurations
with different materials will be considered so that only an acceptable heat
flux can pass the reflector. %porosity
Based on the cooling power of current kryostats it is estimated, that the
transmitted heat flux needs to be in the order of $10\ \si{mW}$.

In contrast to the optical case, the complete spectrum of thermal
radiation for all incident angles $\theta$ and polarisation modes $i$ is
relevant for the total expected heat flow through the reflector
\begin{equation} \label{eq:totalheatflow}
    J_z = \sum\limits_{i=\{L,TV,TH\}}\ \int_0^{\omega_c}\int_{\theta, \phi}
    \mathcal{T}_i(\omega, \theta, \phi)\ g_i(\omega)\ n_B(\omega, T)\
    \hslash \omega\  c_i\cos \theta \dd{\omega} \dd{\theta}\dd{\phi}
\end{equation} \todo{normalisation factor for angle integrations?}
which will be justified later. The integrant
consists of the transmittivity function $\mathcal{T}_i(\omega, \theta, \phi)$
which is defined as proportion of transmitted intensity and can be interpreted
here as transmission probability density. $g_i(\omega)
    n_B(\omega,T)\hslash\omega$ is the spectral energy density per unit volume
of the phononic thermal radiation with Bose distribution $n_B$ and density of
states $g_i$. Factoring this with the phonon velocity in z-direction $c_i\
    \cos\theta$ yields the spectral radiance of the uppermost medium which is
the emitted power per frequency and spatial angle in z-direction. The exact
form of $g_i(\omega)$ and $\omega_c$ is given by the Debye model which can be
applied here.

As qubits are cooled to temperatures of less than $\SI{30}{mK}$, the
temperature of the uppermost medium is also at low Kelvin temperatures. This
justifies using the Debye model for the description of phonons which assumes
linear dispersion $\omega(\uline{k})=c_i |\uline{k}|$. For that reason,
the calculation needs to be cut off at the Debye frequency
$\omega_D = (6\pi^2 n c_i^3)^{1/3}$ where $n$ is the atomic density.
In the following work the upper temperature is assumed to be $T=1.8\si{K}$
which is a temperature that can be achieved comfortably with current methods.
With atomic densities of order $10^{28}\si{\frac{1}{m^3}}$ and sound velocities
of order $10^3\si{\frac{m}{s}}$, the Debye frequency can approximated to be in
the order of $10^{13}\si{s^{-1}}$. The spectral radiance for these frequencies
at $T=1.8\ \si{K}$ is less than a thousandth of its maximum value, so that the
present calculations will be performed with a lower, not material dependent
cutoff of $\omega_c=3\cdot 10^{12}\si{\frac{1}{s}}$.
In this frequency range the resulting wavelengths which are given by
$\lambda= \frac{2\pi c}{\omega}$, are in the order of few nanometers, so that
lattice properties of the materials can be neglected. Also, elastic
anisotropies that may occur in crystal structures are simplified to an
isotropic material with averaged properties.

\todo[inline]{Previous Paragraph: Is this much detail useful in the
    introduction or
    should it be moved to the theory chapter together with the justification
    of \ref{eq:totalheatflow} ?}

\section{Outline}
In the following, the theory neccessary for understanding the model for phonons
in the ballistic regime, namely elastic wave propagation is introduced. Also,
Methods for computing the transmission through a system of layers are presented
and some general properties of such structures are described.
The code structure of the simulation that was written on basis of
that theory is presented afterwards with the aim of providing an overview over 
functionalities and scope of application.