In recent years quantum computing technology has been of rising
interest to scientists and the public. The early steps of exploiting the nature
of quantum states by fabricating and controlling individual qubits  are
overcome \cite{Leonard2019} and computational advantage of quantum processors
over
classical computers, has already been demonstrated in some cases
\cite{Arute2019}. With scalable concepts for many
different implementations \cite{Cirac2000} \cite{Brecht2016} \cite{Bluhm2019}
\cite{Lao2018}	it is only a matter of time until quantum computers can be used
commercially.

However, there are still a some hurdles to overcome before current setups are
at this level. One of them is to reduce the decoherence time. It is crucial for
the execution of long operations or algorithms and strongly
depends on external conditions like, for example, the influence of thermal
radiation consisting of photons or heat conduction from phonons or
electrons\cite{Ferraro2019}.
In solid state qubits, heat transfer by phonons is mostly
enabled by electronics connected to the chip.
The reduction of the decoherence time caused from heat conduction to the qubit
could be solved by an intermediate medium with very low thermal
conductivity shielding the chip from thermal phonons.

\section{Superlattice Structures for Thermal Isolation}
One possible approach to
construct such a medium is a superlattice structure of alternating layers of
different materials whose phononic heat conduction is minimized for coherent
heat phonons. This concept exploits
the properties of coherent phonons modelled as coherent elastic waves and has
proven to be effective for control of heat conduction at low temperatures
\cite{Luckyanova2012}. The underlying theory of this concept focuses on a
special case of the macroscopic theory of heat conduction, where heat flow is
explained as diffusive scattering of phonons between regions of different
temperatures.
Typically the phase of a phonon changes through scattering so that interference
effects are not observable at large scale structures. This means that the
described layer structures need to have thicknesses far smaller than
the phonon mean free path. This case is known as ballistic regime.

Further, such superlattices are known from optics as Distributed Bragg
reflectors and usually consist of layer structures with alternating refraction
indices. These are known to have so called spectral band gaps which are
spectral regions where the wave is fully reflected.
In optics, this structure is usually used as highly efficient reflector for
electromagnetic waves from a defined angle.

In contrast to the optical case, the complete spectrum of thermal
radiation for all incident angles $\theta$ and polarisation modes $i$ is
relevant for the total expected heat flow through the reflector
\begin{equation} \label{eq:totalheatflow}
    J_z = \sum\limits_{i=\{L,TV,TH\}}\ \int_0^{\omega_c}\int_{\theta, \phi}
    \mathcal{T}_i(\omega, \theta, \phi)\ g_i(\omega)\ n_B(\omega, T)\
    \hslash \omega\  c_i\cos \theta \dd{\omega} \dd{\theta}\dd{\phi}
\end{equation}
which will be justified later. The integrand
consists of the transmittivity function $\mathcal{T}_i(\omega, \theta, \phi)$
which is defined as proportion of transmitted intensity and can be interpreted
here as transmission probability density. $g_i(\omega)
    n_B(\omega,T)\hslash\omega$ is the spectral energy density per unit volume
of the phononic thermal radiation with Bose distribution $n_B$ and density of
states $g_i$. Mulitiplying this with the phonon velocity in z-direction $c_i\
    \cos\theta$ yields the spectral radiance of the uppermost medium which is
the emitted power per frequency and spatial angle in z-direction. The exact
form of $g_i(\omega)$ and $\omega_c$ is given by the Debye model which can be
applied here.

Based on the cooling power of current cryostats it is estimated, that the
transmitted heat flux needs to be on the order of $10\ \si{mW}$. Also,
for any power considerations the upper temperature is assumed to be
$T=1.8\si{K}$
which is a temperature that can be achieved comfortably with current methods.

The aim of this work will be to investigate practicable reflector
configurations by developing a reusable simulation of elastic wave propagation
through arbitrary multilayer structures.  %porosity

\section{Previous Work}
The described aim of developing a simulation environment for elastic wave
propagation through multilayer structures has been already pursued by Sebastian
Miles in the scope of a HiWi project for the institute. The level of
developement at the time of my aquisition consisted of the two
implementations for the later described Transfer Matrix Method and the LSE
Method, which could only deliver valid results for cases without mixing of
modes at an interface and also similar numerical issues that were encountered
in this work. In order to avoid adopting those errors, the here described
implementation was written independently.

However, a large part of the theoretical foundations of the described methods
were researched and summed up in a currently not published paper by Miles. The
present theoretical introduction follows it roughly and is worked out with more
detail than it is usual in a paper.

\section{Outline}
In the following, the theory of phononic heat conduction in the ballistic
regime is introduced with focus on explaining the solving methods used in the
developed simulation tool. This requires a detailed consideration of elastic
wave mechanics. Also, the simulation methods for computing the transmission
through a system of layers are derived and some general properties of such
structures are described.
The code structure of the simulation that was written on basis of
that theory is presented afterwards with the aim of providing an overview over
functionalities and scope of application. Further these methods are validated
for single interface scattering and its capabilities are demonstrated on an
exemplary structure.