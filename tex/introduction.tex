\chapter{Introduction}
Initiated by \citet{Benioff1980}, \citet{Feynman1982}, \citet{Deutsch97}, and others in the early 1980s, the field of quantum computing has grown rapidly in the following years, resulting in both a more profound theoretical understanding of semiconductor physics and quantum algorithms as well as growing experimental expertise and sophistication in the fabrication of nano-structures, quantum measurement, and quantum optics. Quantum computing promises significant decreases in computation time for certain problems such as cryptography and integer factorization, the latter having been shown to be possible in polynomial time using Shor's algorithm \cite{Shor1999}.

In 1998, \citet{Loss1998} proposed using the spin states of coupled single-electron quantum dots as the elementary units of quantum computing, the qubit. Following this approach, a popular method encodes qubits in the spin states, generally denoted by $\ket{0}$ for spin down and $\ket{1}$ for spin up, of two adjacent single-electron quantum dots which form a singlet-triplet system. Information is stored in the $S = \left(\ket{10}-\ket{01}\right)/\sqrt{2}$ and $T_{0} = \left(\ket{10}+\ket{01}\right)/\sqrt{2}$ states, where $\ket{ij} = \ket{i}\otimes\ket{j}$, since they have a vanishing magnetic quantum number $m_{s} = 0$ and are therefore unaffected by uniform changes in the magnetic field \cite{Petta2005}. A natural choice for the realization of this kind of qubit are semiconductors.

\section{Spin-Qubits in Semiconductor heterostructures}
Using semiconductor solid state structures as the basis for realizing spin-qubits is a promising approach because of the possible scalability given by the already existing semiconductor electronics industry. There are two predominant approaches: first, using a GaAs/AlGaAs heterostructure to create a two-dimensional electron gas (2DEG) whose potential landscape can be manipulated by gate electrodes on top of the AlGaAs layer, and second, using a silicon-based system such as SiGe or silicon metal-oxide-semiconductor (Si-MOS). While the latter promises longer spin coherence times crucial to high fidelity gates due to, among others, a low nuclear spin density, achieving the same fabrication quality as in GaAs heterostructures is more difficult \cite{DasSarma2011}. \Cref{fig:heterostructure} shows a typical heterostructure design for a GaAs spin quibit.



The spin state in such a setup can be read out by exploiting spin-to-charge conversion due to Pauli spin blockade in the conductance through the quantum dots which can be measured using quantum point contacts (QPC) positioned at the ends of a linear array of quantum dots \cite{Vandersypen2004,Johnson2005,Hanson2007}. Manipulating the exchange interaction $\mathcal{H}_s(t) = J(t)\vec{S}_1\cdot\vec{S}_2$ between the two spins e.g. by adjusting the potential barrier then allows for electrically controlled gate operations \cite{Loss1998}. Commonly, a gate layout features gates that control the tunnel barriers between the quantum dots, gates that control the barriers to the leads, i.e. the reserviors, and \enquote{plunger gates} that control the potential of the dots, that is, in effect the number of electrons. While semiconductor spin qubits present a promising approach to a scalable quantum computer, several problems remain to be solved. For example, a scalable gate layout is yet to be devised and the fabrication of many well-functioning quantum dots is still a challenge.

\section{Tuning Procedures}
\label{sec:intro_tuning}
As previously mentioned, in order to realize a qubit in the heterostructure, ideally, there has to be a \textit{single} electron in each quantum dot. The process of reducing the number of electrons in a dot by means of applying more negative voltages to the plunger gate until it is completely depleted and finally increasing the voltages just enough to let a single electron tunnel through the leads into the dot is part of what is known as \enquote{tuning}. In practice, \enquote{charge stability diagrams} are recorded and plotted as functions of the gate voltages. A popular approach using radio-frequency reflectometry (RF) measures changes in the conductance through the QPCs that is highly sensitive to the charges on the dots \cite{Cassidy2007,Reilly2007}. Since one can't directly obtain the exact number of electrons from the conductance, the information about when the number \textit{changes} is used. This corresponds to jumps in the measuring signal that show as lines in the charge diagram if differentiated, forming in double quantum dots a characteristic honeycomb structure. These are hexagonal areas (also called charge domains) enclosed by the transition lines. Each line represents a transition between different occupancy states. Thus, once no more lines show up in the charge diagram the zero-electron regime has been reached and the plunger voltages can be increased until the first transition line is crossed again such that there is only one electron in each dot.

For a growing number of qubits and by that quantum dots, the tuning process becomes increasingly complex and the charge diagrams arduous to understand and analyze since the dot occupancy is a function of the $n$ plunger gate voltages and, to a lesser extent, other gates as well, creating an at least $n$-dimensional parameter space. While distant dots will couple very weakly and thus can be treated as independent of each other, by that diagonalizing the parameter space somewhat, it is still a considerable complication. It is therefore conducive to simulate charge diagrams for the given system in order to get a better understanding of the effects of the voltages applied. Moreover, several QPCs are used for readout, increasing the number of measurement signals. As experimental capability in realizing multiple coupled quantum dots advances and with scalability of quantum computing in mind, the automation of quantum dot tuning is therefore of growing interest. The goal of this thesis is to take a first step towards a fully computer-automated tuning process by concentrating on the coarse tuning of the system to the few-electron regime.

\section{Previous Approaches}
The majority of research effort in the field of semiconductor spin qubits in the last years has focused on the realization, fabrication, and characterization of semiconductor qubits, while little attention has gone into the automation of the tuning process since for double quantum dots (DQD) -- until recently state of the art -- manual tuning of the dots is still feasible if tedious. Recently, though, \citet{Baart2016} have demonstrated computer automated tuning of a double-dot GaAs system. They focus on tuning the sensing dots and detect triple-points, namely where the transition lines from two dots meet, by matching a reference image to regions in the charge diagram and thresholding the response to the fit.

\citet{Botzem2016} present methods complementary to the results by \citeauthor{Baart2016} by focusing on the fine-tuning of the dots once the single-electron regime has been reached, i.e. the characterization of tunnel couplings and the location \enquote{of various fast initialization points for the qubit}.

\citet{Delbecq2014} report the realization of a quadruple quantum dot (QQD) array in the single-electron regime and apply a capacitive model with good agreement to the experiment while \citet{Bogan2016} give a survey of tuning strategies for linear QQD arrays. They find that the most efficient tuning method for their layout with two leads located at the sides of the linear array is to first create a single large dot, then split it in two, tune both resulting dots to the few-electron regime, and finally splitting each dot again into two separate ones. This way they avoid charge hysteresis observed in other approaches.

\section{Assumptions}
The work at hand is primarily concerned with the simulation of charge diagrams for multiple quantum dots, methods for automated line detection in said diagrams, and a tentative proposal for the process of tuning the system to the zero-electron regime applied to double-dot diagrams. However, it should be made clear that \enquote{tuning} here by no means refers to a complete description of the actual tuning process, which requires also the tuning and re-tuning of the sensing dots, the tunnel couplings, and more. Thus, in the following I am going to assume that these components of the tuning process are well-implemented already and readily available.

\section{Outline}
In \cref{chap:simulation} I will describe the simulation of multi-dot charge diagrams. \Cref{sec:classical_description} explains the classical capacitive model used to calculate the energy landscape. \Cref{sec:cd_implementation} describes the implementation of the model in \matlab{}. I analyze DQD charge diagrams in \cref{chap:analysis} with tuning in mind. First, I motivate the approach in the context of tuning, then I give a survey of available methods for line detection implemented in \matlab{} in \cref{sec:line_detection_matlab}, and introduce a line detection algorithm making use of the Radon transform in \cref{sec:radon,sec:detection_implementation}. Finally, I apply the algorithm to charge diagrams both simulated and real in \cref{sec:application_simulated,sec:application_real}. Lastly, I present an algorithm that attempts to tune a double-dot system to the zero-electron regime in \cref{sec:tuning_scheme} and apply it to simulated double quantum dot charge diagrams in \cref{sec:tuning_application}.
